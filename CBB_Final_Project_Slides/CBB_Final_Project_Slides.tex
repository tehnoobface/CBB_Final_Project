\documentclass{beamer}
\usetheme{Madrid} % Choose your favorite theme here

% Packages
\usepackage{graphicx}
\usepackage{amsmath}
\usepackage{caption}
\usepackage{subcaption}

% Title page
\title[NCAA Basketball Analysis]{NCAA Basketball Analysis}
\author[STAT 386]{Maclean Sherren \& Kayla Tansiongco}
\institute[Brigham Young University]{Brigham Young University}
\date{\today}

\begin{document}

\begin{frame}
\titlepage
\end{frame}

\begin{frame}{Outline}
\tableofcontents
\end{frame}

\section{Introduction}

\begin{frame}{Introduction}
\begin{itemize}
  \item Due to Covid, 2020 NCAA tournament did not happen.
  \item Question: Are there certain regular season variables that are better predictors for postseason outcomes?
  \item Goal: Use existing data/trends from other years and regular season data from 2020 to come up with potential methods of prediction and postseason outcomes.
\end{itemize}
\end{frame}

\section{Data Collection}

\begin{frame}{Data Collection}
\begin{itemize}
  \item College Basketball Dataset folder found on Kaggle. Dataset "cbb.csv" has regular and postseason statistics from years 2013-2021. Excludes 2020 season due to lack of postseason statistics.
  \item Folder included datasets for each individual year, with 2020 ("cbb20.csv") having regular season stats only.
  \item Variables of final dataset: YEAR, TEAM, CONF, G, W, BARTHAG, \textcolor{blue}{POSTSEASON}, \textcolor{blue}{SEASON\_FINAL}.
\end{itemize}
\end{frame}

\begin{frame}{Data Collection}
\begin{itemize}
  \item Created numerically coded column, SEASON\_FINAL, derived from POSTSEASON (string data type) for cbb.csv dataframe. (Ex: "Champions" = 1, "2ND" = 2, "F4" = 4)
  \item Initialized variables POSTSEASON and SEASON\_FINAL as NaN for cbb20.csv dataframe. This is what we will fill in through our prediction methods.
\end{itemize}
\end{frame}

\section{Exploratory Data Analysis}

\begin{frame}{Exploratory Data Analysis}
\begin{itemize}
  \item Determine what variables are most correlated with POSTSEASON outcomes.
  \item Used Spearman correlation since it is more appropriate for measurements taken from ordinal scales.
\end{itemize}
\end{frame}

\begin{frame}{Exploratory Data Analysis}
  \begin{center}
    \begin{itemize}
        \item BARTHAG vs SEASON\_FINAL
        \item BARTHAG = Power Rating (Chance of beating an average Division 1 team)
    \end{itemize}

    \begin{figure}
      \centering
      \includegraphics[width=0.5\linewidth]{barthagscatterplot.png} % Replace with barthagscatterplot
    \end{figure}

    \vspace{0.1cm}

    Corr: -0.6461273899341722
  \end{center}
\end{frame}

\begin{frame}{Exploratory Data Analysis}
  \begin{center}
    \begin{itemize}
        \item W vs SEASON\_FINAL
        \item W = Number of games won
    \end{itemize}

    \begin{figure}
      \centering
      \includegraphics[width=0.5\linewidth]{wscatterplot.png} % Replace with w scatterplot
    \end{figure}

    \vspace{0.1cm}

    Corr: -0.5125436890870254
  \end{center}
\end{frame}

\begin{frame}{Exploratory Data Analysis}
  \begin{center}
    \begin{itemize}
        \item W/L vs SEASON\_FINAL
        \item W/L = Win-Loss Ratio
    \end{itemize}

    \begin{figure}
      \centering
      \includegraphics[width=0.5\linewidth]{wlscatterplot.png} % Replace with w/l scatterplot
    \end{figure}

    \vspace{0.1cm}

    Corr: -0.3532346489556209

  \end{center}
\end{frame}

\begin{frame}{Exploratory Data Analysis}
  \begin{center}
    \begin{itemize}
        \item SEED vs SEASON\_FINAL
        \item SEED = Seed in the NCAA March Madness Tournament
    \end{itemize}

    \begin{figure}
      \centering
      \includegraphics[width=0.5\linewidth]{seedscatterplot.png} % Replace with seed scatterplot
    \end{figure}

    \vspace{0.1cm}

    Corr: 0.6065146756151623
  \end{center}
\end{frame}

\section{Methodology}

\begin{frame}{Methodology}
\begin{itemize}
  \item Package contains a function that takes a season and method of prediction, then produces simulated results of an NCAA tournament.
  \item Compare predicted results to actual results.
  \item Create a prediction for 2020 NCAA postseason tournament.
\end{itemize}
\end{frame}

\begin{frame}{Methodology}
\begin{itemize}
  \item Season: 2021
  \item Method: BARTHAG (Power Rating)
\end{itemize}

\begin{columns}[T] % Align columns at the top
\begin{column}{0.5\textwidth} % Left column
  \textbf{Predicted Final Four}
  \begin{itemize}
    \item Baylor
    \item Michigan
    \item Gonzaga
    \item Houston
  \end{itemize}
  \textbf{Predicted Championship}
  \begin{itemize}
    \item Gonzaga vs Baylor
    \item Winner: Baylor
    \end{itemize}
\end{column}
\begin{column}{0.5\textwidth} % Right column
  \textbf{Actual Final Four}
  \begin{itemize}
    \item Baylor
    \item Gonzaga
    \item Houston
    \item UCLA
  \end{itemize}
  \textbf{Actual Championship}
  \begin{itemize}
  \item Gonzaga vs Baylor
  \item Winner: Baylor
  \end{itemize}
\end{column}
\end{columns}
\end{frame}

\begin{frame}{Methodology}
\begin{itemize}
  \item Season: 2021
  \item Method: W (Wins)
\end{itemize}

\begin{columns}[T] % Align columns at the top
\begin{column}{0.5\textwidth} % Left column
  \textbf{Predicted Final Four}
  \begin{itemize}
    \item Winthrop
    \item Alabama
    \item Gonzaga
    \item Loyola Chicago
  \end{itemize}
  \textbf{Predicted Championship}
  \begin{itemize}
    \item Alabama vs Gonzaga
    \item Winner: Alabama
    \end{itemize}
\end{column}
\begin{column}{0.5\textwidth} % Right column
  \textbf{Actual Final Four}
  \begin{itemize}
    \item Baylor
    \item Gonzaga
    \item Houston
    \item UCLA
  \end{itemize}
  \textbf{Actual Championship}
  \begin{itemize}
    \item Gonzaga vs Baylor
    \item Winner: Baylor
    \end{itemize}
\end{column}
\end{columns}
\end{frame}

\begin{frame}{Methodology}
\begin{itemize}
  \item Season: 2021
  \item Method: SEED
\end{itemize}

\begin{columns}[T] % Align columns at the top
\begin{column}{0.5\textwidth} % Left column
  \textbf{Predicted Final Four}
  \begin{itemize}
    \item Baylor
    \item Gonzaga
    \item Michigan
    \item Illinois
  \end{itemize}
  \textbf{Predicted Championship}
  \begin{itemize}
    \item Michigan vs Gonzaga
    \item Winner: Gonzaga
    \end{itemize}
\end{column}
\begin{column}{0.5\textwidth} % Right column
  \textbf{Actual Final Four}
  \begin{itemize}
    \item Baylor
    \item Gonzaga
    \item Houston
    \item UCLA
  \end{itemize}
  \textbf{Actual Championship}
  \begin{itemize}
    \item Gonzaga vs Baylor
    \item Winner: Baylor
    \end{itemize}
\end{column}
\end{columns}
\end{frame}

\section{Results}

\begin{frame}{Results}
\begin{itemize}
  \item Season: 2020 
  \item Method: BARTHAG (Power Rating)
  
      \begin{figure}
      \centering
      \includegraphics[width=0.7\linewidth]{example-image} % Replace with barthag method results
    \end{figure}
    
\end{itemize}
\end{frame}

\begin{frame}{Results}
\begin{itemize}
  \item Season: 2020 
  \item Method: W (Wins)
  
      \begin{figure}
      \centering
      \includegraphics[width=0.7\linewidth]{example-image} % Replace with w method results
    \end{figure}
    
\end{itemize}
\end{frame}

\begin{frame}{Results}
\begin{itemize}
  \item Season: 2020 
  \item Method: RK (Rank/Equivalent to SEED)
  
      \begin{figure}
      \centering
      \includegraphics[width=0.7\linewidth]{example-image} % Replace with seed method results
    \end{figure}
    
\end{itemize}
\end{frame}

\section{Conclusion}

\begin{frame}{Conclusion}
\begin{itemize}
  \item Summarize the main findings and their implications.
  \item Reflect on the success of your project and any limitations.
  \item Suggest areas for future research or improvements.
\end{itemize}
\end{frame}

\section{Q \& A}

\begin{frame}{Questions \& Answers}
\begin{center}
\Huge Questions or comments?
\end{center}
\end{frame}

\begin{frame}{Thank You}
\begin{center}
\Huge Thank you! Happy Holidays!
\end{center}
\end{frame}

\end{document}
